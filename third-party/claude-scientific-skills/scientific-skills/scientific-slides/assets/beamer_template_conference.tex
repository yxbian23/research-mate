\documentclass[aspectratio=169,11pt]{beamer}

% Encoding
\usepackage[utf8]{inputenc}
\usepackage[T1]{fontenc}

% Theme and colors
\usetheme{Madrid}
\usecolortheme{beaver}

% Remove navigation symbols
\setbeamertemplate{navigation symbols}{}

% Page numbers in footer
\setbeamertemplate{footline}[frame number]

% Graphics
\usepackage{graphicx}
\graphicspath{{./figures/}}

% Math
\usepackage{amsmath, amssymb}

% Tables
\usepackage{booktabs}

% Citations
\usepackage[style=authoryear,maxcitenames=2,backend=biber]{biblatex}
\addbibresource{references.bib}
\renewcommand*{\bibfont}{\tiny}

% Colors (customize these)
\definecolor{primaryblue}{RGB}{0,90,156}
\definecolor{secondaryorange}{RGB}{228,108,10}

% Custom colors for theme elements
\setbeamercolor{structure}{fg=primaryblue}
\setbeamercolor{title}{fg=primaryblue}
\setbeamercolor{frametitle}{fg=primaryblue}
\setbeamercolor{block title}{fg=white,bg=primaryblue}

% Title page information
\title[Short Title]{Full Presentation Title:\\Descriptive and Specific}
\subtitle{Optional Subtitle}
\author[Author Name]{Author Name\inst{1}}
\institute[Institution]{
  \inst{1}
  Department of XYZ\\
  University Name\\
  \vspace{0.2cm}
  \texttt{email@university.edu}
}
\date{Conference Name\\Month Day, Year}

% Optional: Logo
% \logo{\includegraphics[height=0.8cm]{logo.png}}

\begin{document}

% Title slide
\begin{frame}[plain]
  \titlepage
\end{frame}

% Outline (optional for conference talks)
% \begin{frame}{Outline}
%   \tableofcontents
% \end{frame}

%==============================================
% INTRODUCTION
%==============================================

\section{Introduction}

\begin{frame}{The Problem}
  \begin{itemize}
    \item<1-> Start with a compelling hook or problem statement
    \item<2-> Establish why this research matters
    \item<3-> Set up the knowledge gap
    \item<4-> Preview your contribution
  \end{itemize}
  
  \vfill
  
  \uncover<4->{
    \begin{block}{Research Question}
      State your specific research question or hypothesis clearly
    \end{block}
  }
\end{frame}

\begin{frame}{Background and Context}
  \begin{columns}[T]
    
    \begin{column}{0.5\textwidth}
      \textbf{Prior Work:}
      \begin{itemize}
        \item Key finding 1 \cite{reference1}
        \item Key finding 2 \cite{reference2}
        \item Knowledge gap identified
      \end{itemize}
    \end{column}
    
    \begin{column}{0.5\textwidth}
      % Example figure
      \begin{figure}
        \centering
        % \includegraphics[width=\textwidth]{context_figure.pdf}
        \framebox[0.9\textwidth][c]{[Figure: Context or Prior Work]}
        \caption{Illustration of the problem}
      \end{figure}
    \end{column}
    
  \end{columns}
\end{frame}

%==============================================
% METHODS
%==============================================

\section{Methods}

\begin{frame}{Study Design}
  \begin{columns}[T]
    
    \begin{column}{0.6\textwidth}
      \textbf{Approach:}
      \begin{itemize}
        \item Study type/design
        \item Participants/sample (n = X)
        \item Key procedures
        \item Analysis strategy
      \end{itemize}
      
      \vspace{0.5cm}
      
      \begin{alertblock}{Key Innovation}
        Highlight what makes your approach novel or improved
      \end{alertblock}
    \end{column}
    
    \begin{column}{0.4\textwidth}
      \begin{figure}
        \centering
        % \includegraphics[width=\textwidth]{methods_schematic.pdf}
        \framebox[0.9\textwidth][c]{[Methods Diagram]}
        \caption{Experimental design}
      \end{figure}
    \end{column}
    
  \end{columns}
\end{frame}

\begin{frame}{Analysis Overview}
  \begin{itemize}
    \item \textbf{Primary outcome:} What you measured
    \item \textbf{Statistical approach:} Tests used
    \item \textbf{Sample size justification:} Power analysis (if applicable)
    \item \textbf{Software:} Tools and versions used
  \end{itemize}
  
  \vspace{0.5cm}
  
  % Optional: Show key equation
  \begin{exampleblock}{Key Model}
    \begin{equation}
      Y = \beta_0 + \beta_1 X_1 + \beta_2 X_2 + \epsilon
    \end{equation}
  \end{exampleblock}
\end{frame}

%==============================================
% RESULTS
%==============================================

\section{Results}

\begin{frame}{Main Finding 1}
  \begin{figure}
    \centering
    % \includegraphics[width=0.85\textwidth]{result1.pdf}
    \framebox[0.8\textwidth][c]{[Figure: Main Result 1]}
    \caption{Primary outcome showing significant effect ($p < 0.001$)}
  \end{figure}
  
  \vspace{0.3cm}
  
  \begin{itemize}
    \item<2-> Key observation: Description of pattern
    \item<3-> Statistical result: Effect size and significance
    \item<4-> Interpretation: What this means
  \end{itemize}
\end{frame}

\begin{frame}{Main Finding 2}
  \begin{columns}[c]
    
    \begin{column}{0.5\textwidth}
      \begin{figure}
        \centering
        % \includegraphics[width=\textwidth]{result2a.pdf}
        \framebox[0.9\textwidth][c]{[Result 2A]}
        \caption{Condition A}
      \end{figure}
    \end{column}
    
    \begin{column}{0.5\textwidth}
      \begin{figure}
        \centering
        % \includegraphics[width=\textwidth]{result2b.pdf}
        \framebox[0.9\textwidth][c]{[Result 2B]}
        \caption{Condition B}
      \end{figure}
    \end{column}
    
  \end{columns}
  
  \vspace{0.5cm}
  
  \begin{itemize}
    \item Comparison shows: Key difference
    \item Statistical test: $t(50) = 3.4, p = 0.001$
  \end{itemize}
\end{frame}

\begin{frame}{Supporting Evidence}
  \begin{table}
    \centering
    \caption{Summary of key results across conditions}
    \begin{tabular}{lccc}
      \toprule
      \textbf{Condition} & \textbf{Metric 1} & \textbf{Metric 2} & \textbf{$p$-value} \\
      \midrule
      Control    & 45.2 $\pm$ 3.1 & 0.65 & --- \\
      Treatment  & 67.8 $\pm$ 2.9 & 0.82 & $< 0.001$ \\
      \bottomrule
    \end{tabular}
  \end{table}
  
  \vspace{0.5cm}
  
  \begin{itemize}
    \item Consistent pattern across multiple metrics
    \item Effect robust to various controls
  \end{itemize}
\end{frame}

%==============================================
% DISCUSSION
%==============================================

\section{Discussion}

\begin{frame}{Interpretation}
  \textbf{Key Findings:}
  \begin{enumerate}
    \item First main result and its significance
    \item Second main result and its implications
    \item Supporting evidence strengthens conclusions
  \end{enumerate}
  
  \vspace{0.5cm}
  
  \textbf{Relation to Prior Work:}
  \begin{itemize}
    \item Consistent with \cite{reference1}
    \item Extends beyond \cite{reference2}
    \item Resolves controversy from \cite{reference3}
  \end{itemize}
\end{frame}

\begin{frame}{Implications and Impact}
  \begin{columns}[T]
    
    \begin{column}{0.5\textwidth}
      \textbf{Scientific Impact:}
      \begin{itemize}
        \item Advances understanding of X
        \item Provides new framework for Y
        \item Opens avenue for Z research
      \end{itemize}
    \end{column}
    
    \begin{column}{0.5\textwidth}
      \textbf{Practical Applications:}
      \begin{itemize}
        \item Clinical relevance
        \item Policy implications
        \item Technological applications
      \end{itemize}
    \end{column}
    
  \end{columns}
  
  \vspace{0.5cm}
  
  \begin{block}{Limitations}
    \begin{itemize}
      \item Acknowledge key limitation 1
      \item Note limitation 2 and how future work addresses it
    \end{itemize}
  \end{block}
\end{frame}

%==============================================
% CONCLUSION
%==============================================

\section{Conclusion}

\begin{frame}{Conclusions}
  \begin{block}{Key Takeaways}
    \begin{enumerate}
      \item \textbf{First main finding:} Brief statement
      \item \textbf{Second main finding:} Brief statement
      \item \textbf{Broader impact:} Significance for field
    \end{enumerate}
  \end{block}
  
  \vspace{0.5cm}
  
  \textbf{Future Directions:}
  \begin{itemize}
    \item Extend to population/context Y
    \item Investigate mechanism Z
    \item Collaborate with domain X
  \end{itemize}
\end{frame}

\begin{frame}[plain]
  \begin{center}
    {\Large \textbf{Thank You}}
    
    \vspace{1cm}
    
    {\large Questions?}
    
    \vspace{1cm}
    
    \begin{columns}
      \begin{column}{0.5\textwidth}
        \textbf{Contact:}\\
        Author Name\\
        \texttt{email@university.edu}\\
        \url{https://yourwebsite.edu}
      \end{column}
      
      \begin{column}{0.5\textwidth}
        % Optional: QR code to paper or website
        % \includegraphics[width=3cm]{qrcode.png}\\
        % {\small Scan for paper/code}
      \end{column}
    \end{columns}
    
    \vspace{0.5cm}
    
    {\footnotesize
      Funding: Grant Agency Award \#12345\\
      Collaborators: Colleague 1, Colleague 2
    }
  \end{center}
\end{frame}

%==============================================
% BACKUP SLIDES
%==============================================

\appendix

\begin{frame}{Backup: Additional Data}
  \begin{figure}
    \centering
    % \includegraphics[width=0.7\textwidth]{supplementary_figure.pdf}
    \framebox[0.6\textwidth][c]{[Supplementary Analysis]}
    \caption{Additional analysis for questions}
  \end{figure}
\end{frame}

\begin{frame}{Backup: Methodological Details}
  \textbf{Detailed Procedure:}
  \begin{itemize}
    \item Step-by-step protocol details
    \item Equipment specifications
    \item Parameter settings
    \item Quality control measures
  \end{itemize}
  
  \vspace{0.5cm}
  
  \textbf{Alternative Analyses:}
  \begin{itemize}
    \item Sensitivity analysis results
    \item Different statistical approaches
    \item Subgroup analyses
  \end{itemize}
\end{frame}

%==============================================
% REFERENCES
%==============================================

\begin{frame}[allowframebreaks]{References}
  \printbibliography
\end{frame}

\end{document}

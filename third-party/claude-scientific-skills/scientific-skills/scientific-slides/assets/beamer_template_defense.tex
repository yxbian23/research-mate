\documentclass[aspectratio=169,12pt]{beamer}

% Encoding
\usepackage[utf8]{inputenc}
\usepackage[T1]{fontenc}

% Theme - professional and formal for defense
\usetheme{Boadilla}
\usecolortheme{whale}

% Remove navigation symbols
\setbeamertemplate{navigation symbols}{}

% Page numbers with total
\setbeamertemplate{footline}{
  \leavevmode%
  \hbox{%
  \begin{beamercolorbox}[wd=.333333\paperwidth,ht=2.25ex,dp=1ex,center]{author in head/foot}%
    \usebeamerfont{author in head/foot}\insertshortauthor
  \end{beamercolorbox}%
  \begin{beamercolorbox}[wd=.333333\paperwidth,ht=2.25ex,dp=1ex,center]{title in head/foot}%
    \usebeamerfont{title in head/foot}\insertshorttitle
  \end{beamercolorbox}%
  \begin{beamercolorbox}[wd=.333333\paperwidth,ht=2.25ex,dp=1ex,right]{date in head/foot}%
    \usebeamerfont{date in head/foot}\insertshortdate{}\hspace*{2em}
    \insertframenumber{} / \inserttotalframenumber\hspace*{2ex}
  \end{beamercolorbox}}%
  \vskip0pt%
}

% Section pages
\AtBeginSection[]{
  \begin{frame}
    \vfill
    \centering
    \begin{beamercolorbox}[sep=8pt,center,shadow=true,rounded=true]{title}
      \usebeamerfont{title}\insertsectionhead\par%
    \end{beamercolorbox}
    \vfill
  \end{frame}
}

% Graphics
\usepackage{graphicx}
\graphicspath{{./figures/}}

% Math
\usepackage{amsmath, amssymb, amsthm}

% Tables
\usepackage{booktabs}
\usepackage{multirow}

% Citations
\usepackage[style=authoryear,maxcitenames=2,backend=biber]{biblatex}
\addbibresource{references.bib}
\renewcommand*{\bibfont}{\scriptsize}

% Custom colors - conservative for formal defense
\definecolor{universityblue}{RGB}{0,60,113}
\definecolor{accentgold}{RGB}{179,136,12}

\setbeamercolor{structure}{fg=universityblue}
\setbeamercolor{title}{fg=universityblue}
\setbeamercolor{frametitle}{fg=universityblue}
\setbeamercolor{block title}{fg=white,bg=universityblue}

% Title page information
\title[Dissertation Defense]{Title of Your Dissertation:\\Comprehensive and Descriptive}
\subtitle{Dissertation Defense}
\author[Your Name]{Your Name, M.S.\\
  \vspace{0.3cm}
  Doctoral Candidate\\
  Department of Your Field}
\institute[University]{
  University Name\\
  \vspace{0.3cm}
  \textbf{Dissertation Committee:}\\
  Prof. Advisor Name (Chair)\\
  Prof. Committee Member 2\\
  Prof. Committee Member 3\\
  Prof. Committee Member 4\\
  Prof. External Member
}
\date{\today}

% University logo
% \logo{\includegraphics[height=0.8cm]{university_logo.png}}

\begin{document}

% Title slide
\begin{frame}[plain]
  \titlepage
\end{frame}

% Committee and acknowledgments
\begin{frame}{Dissertation Committee}
  \begin{center}
    \textbf{Committee Chair:}\\
    Prof. Advisor Name, PhD\\
    Department of Your Field
    
    \vspace{0.5cm}
    
    \textbf{Committee Members:}\\
    Prof. Member 2, PhD -- Department of Related Field\\
    Prof. Member 3, PhD -- Department of Your Field\\
    Prof. Member 4, PhD -- Department of Statistics\\
    Prof. External Member, PhD -- External Institution
    
    \vspace{0.8cm}
    
    \textit{Thank you to my committee for your guidance, support, and invaluable feedback throughout this dissertation research.}
  \end{center}
\end{frame}

% Overview
\begin{frame}{Dissertation Overview}
  \begin{exampleblock}{Central Thesis}
    Brief statement of the overarching thesis or argument that ties together all dissertation studies.
  \end{exampleblock}
  
  \vspace{0.5cm}
  
  \textbf{Dissertation Structure:}
  \begin{itemize}
    \item \textbf{Chapter 1:} Introduction and theoretical framework
    \item \textbf{Chapter 2:} Study 1 -- [Brief description]
    \item \textbf{Chapter 3:} Study 2 -- [Brief description]
    \item \textbf{Chapter 4:} Study 3 -- [Brief description]
    \item \textbf{Chapter 5:} General discussion and conclusions
  \end{itemize}
\end{frame}

\begin{frame}{Outline}
  \tableofcontents
\end{frame}

%==============================================
% CHAPTER 1: INTRODUCTION
%==============================================

\section{Chapter 1: Introduction and Background}

\begin{frame}{The Problem}
  \begin{columns}[T]
    
    \begin{column}{0.5\textwidth}
      \textbf{Real-World Significance:}
      \begin{itemize}
        \item Prevalence: X affects Y million people
        \item Impact: Costs \$Z billion annually
        \item Need: Current solutions inadequate
        \item Opportunity: New approach needed
      \end{itemize}
    \end{column}
    
    \begin{column}{0.5\textwidth}
      \begin{figure}
        \centering
        % \includegraphics[width=\textwidth]{problem_figure.pdf}
        \framebox[0.9\textwidth][c]{[Problem Illustration]}
        \caption{Visualization of the problem}
      \end{figure}
    \end{column}
    
  \end{columns}
  
  \vspace{0.5cm}
  
  \begin{alertblock}{Central Question}
    How can we understand and address this critical challenge using novel theoretical framework X?
  \end{alertblock}
\end{frame}

\subsection{Theoretical Framework}

\begin{frame}{Theoretical Background}
  \textbf{Historical Development:}
  \begin{itemize}
    \item \textbf{Early theories (1950s-1980s):} Established foundational concepts \cite{foundational1975}
    \item \textbf{Modern frameworks (1990s-2000s):} Refined understanding \cite{refinement2000}
    \item \textbf{Recent advances (2010s-present):} Novel approaches emerge \cite{recent2018}
  \end{itemize}
  
  \vspace{0.5cm}
  
  \textbf{Key Theoretical Constructs:}
  \begin{enumerate}
    \item \textbf{Construct A:} Describes mechanism X
    \item \textbf{Construct B:} Explains process Y
    \item \textbf{Construct C:} Predicts outcome Z
  \end{enumerate}
  
  \vspace{0.5cm}
  
  \begin{block}{Theoretical Gap}
    Existing theories fail to account for interaction between A and B under conditions C
  \end{block}
\end{frame}

\begin{frame}{Literature Review: What We Know}
  \begin{columns}[T]
    
    \begin{column}{0.5\textwidth}
      \textbf{Established Findings:}
      \begin{itemize}
        \item Finding 1: Well-replicated
        \item Finding 2: Meta-analytically supported
        \item Finding 3: Cross-culturally validated
        \item Finding 4: Mechanism partially understood
      \end{itemize}
      
      \vspace{0.3cm}
      
      \textbf{Methodological Advances:}
      \begin{itemize}
        \item Technique A: Improved measurement
        \item Technique B: Better controls
        \item Technique C: Novel analysis
      \end{itemize}
    \end{column}
    
    \begin{column}{0.5\textwidth}
      \textbf{Remaining Questions:}
      \begin{itemize}
        \item[\alert{?}] How does A interact with B?
        \item[\alert{?}] What role does C play?
        \item[\alert{?}] Does effect generalize to D?
        \item[\alert{?}] What are boundary conditions?
      \end{itemize}
      
      \vspace{0.3cm}
      
      \begin{exampleblock}{Dissertation Focus}
        This dissertation addresses these gaps through three complementary studies
      \end{exampleblock}
    \end{column}
    
  \end{columns}
\end{frame}

\subsection{Dissertation Aims}

\begin{frame}{Overarching Goals and Specific Aims}
  \begin{block}{Overall Dissertation Goal}
    To develop and test a comprehensive framework for understanding how X influences Y through mechanisms A, B, and C across contexts.
  \end{block}
  
  \vspace{0.5cm}
  
  \textbf{Specific Aims:}
  
  \begin{enumerate}
    \item \textbf{Study 1:} Establish relationship between X and Y
    \begin{itemize}
      \item Method: Cross-sectional survey (n = 500)
      \item Goal: Characterize X→Y relationship
    \end{itemize}
    
    \item \textbf{Study 2:} Identify mediating mechanisms A and B
    \begin{itemize}
      \item Method: Longitudinal study (n = 250, 3 waves)
      \item Goal: Test mediation and temporal precedence
    \end{itemize}
    
    \item \textbf{Study 3:} Test causal model and generalizability
    \begin{itemize}
      \item Method: Experimental manipulation (n = 180)
      \item Goal: Establish causality and boundary conditions
    \end{itemize}
  \end{enumerate}
\end{frame}

%==============================================
% CHAPTER 2: STUDY 1
%==============================================

\section{Chapter 2: Study 1}

\begin{frame}{Study 1: Overview}
  \begin{columns}[T]
    
    \begin{column}{0.5\textwidth}
      \textbf{Research Question:}\\
      Does X predict Y, and is this relationship moderated by individual difference Z?
      
      \vspace{0.5cm}
      
      \textbf{Hypotheses:}
      \begin{enumerate}
        \item H1: X positively predicts Y
        \item H2: Z moderates X→Y
        \item H3: Effect varies by demographic factors
      \end{enumerate}
    \end{column}
    
    \begin{column}{0.5\textwidth}
      \textbf{Design:}
      \begin{itemize}
        \item Cross-sectional survey
        \item N = 500 participants
        \item Online recruitment
        \item Power: .95 for medium effects
      \end{itemize}
      
      \vspace{0.3cm}
      
      \textbf{Measures:}
      \begin{itemize}
        \item X: Validated scale (α = .89)
        \item Y: Performance measure
        \item Z: Individual difference
        \item Controls: Demographics
      \end{itemize}
    \end{column}
    
  \end{columns}
\end{frame}

\begin{frame}{Study 1: Methods}
  \textbf{Participants:}
  \begin{itemize}
    \item N = 500 (62\% female; Age: $M = 34.2$, $SD = 11.5$)
    \item Recruited via university participant pool and online platforms
    \item Inclusion: Ages 18-65, fluent in English
    \item Exclusion: Prior participation in related studies
  \end{itemize}
  
  \vspace{0.5cm}
  
  \textbf{Procedure:}
  \begin{enumerate}
    \item Informed consent and demographics
    \item Battery of questionnaires (45 minutes)
    \item Debriefing and compensation
  \end{enumerate}
  
  \vspace{0.5cm}
  
  \textbf{Analysis:}
  \begin{itemize}
    \item Hierarchical regression for H1 and H2
    \item Moderation analysis using PROCESS macro
    \item Subgroup analyses for H3
  \end{itemize}
\end{frame}

\begin{frame}{Study 1: Results}
  \begin{columns}[c]
    
    \begin{column}{0.6\textwidth}
      \begin{figure}
        \centering
        % \includegraphics[width=\textwidth]{study1_main_result.pdf}
        \framebox[0.9\textwidth][c]{[Study 1: Main Result]}
        \caption{X predicts Y ($\beta = 0.47$, $p < .001$, $R^2 = .22$)}
      \end{figure}
    \end{column}
    
    \begin{column}{0.4\textwidth}
      \textbf{Key Findings:}
      \begin{itemize}
        \item H1 supported: Strong X→Y relationship
        \item H2 supported: Z moderates effect
        \item H3 partially supported: Age effects found
      \end{itemize}
      
      \vspace{0.5cm}
      
      \begin{block}{Conclusion}
        Study 1 establishes foundational X→Y relationship
      \end{block}
    \end{column}
    
  \end{columns}
\end{frame}

%==============================================
% CHAPTER 3: STUDY 2
%==============================================

\section{Chapter 3: Study 2}

\begin{frame}{Study 2: Overview}
  \begin{exampleblock}{Research Question}
    What mechanisms (A and B) mediate the X→Y relationship, and what is the temporal ordering?
  \end{exampleblock}
  
  \vspace{0.5cm}
  
  \textbf{Rationale:}
  \begin{itemize}
    \item Study 1 showed X→Y relationship exists
    \item Need to identify mediating processes
    \item Longitudinal design establishes temporal precedence
    \item Tests proposed theoretical model
  \end{itemize}
  
  \vspace{0.5cm}
  
  \textbf{Design:}
  \begin{itemize}
    \item Three-wave longitudinal study
    \item N = 250, assessments 6 months apart
    \item Measures: X (T1), A and B (T2), Y (T3)
    \item Analysis: Cross-lagged panel model, mediation
  \end{itemize}
\end{frame}

\begin{frame}{Study 2: Methods}
  \begin{columns}[T]
    
    \begin{column}{0.5\textwidth}
      \textbf{Sample:}
      \begin{itemize}
        \item N = 250 at baseline
        \item Retention: 88\% at T2, 82\% at T3
        \item Age: $M = 36.4$, $SD = 12.1$
        \item 58\% female, diverse sample
      \end{itemize}
      
      \vspace{0.5cm}
      
      \textbf{Timeline:}
      \begin{itemize}
        \item T1 (baseline): X measured
        \item T2 (+6 months): A, B measured
        \item T3 (+12 months): Y measured
      \end{itemize}
    \end{column}
    
    \begin{column}{0.5\textwidth}
      \begin{figure}
        \centering
        % \includegraphics[width=\textwidth]{study2_design.pdf}
        \framebox[0.9\textwidth][c]{[Longitudinal Design]}
        \caption{Three-wave design with proposed mediation model}
      \end{figure}
    \end{column}
    
  \end{columns}
  
  \vspace{0.5cm}
  
  \textbf{Analysis:}
  \begin{itemize}
    \item Structural equation modeling for mediation
    \item Cross-lagged panel model for temporal precedence
    \item Missing data handled via FIML
  \end{itemize}
\end{frame}

\begin{frame}{Study 2: Results}
  \begin{figure}
    \centering
    % \includegraphics[width=0.8\textwidth]{study2_mediation.pdf}
    \framebox[0.75\textwidth][c]{[Mediation Model with Path Coefficients]}
    \caption{Serial mediation: X → A → B → Y}
  \end{figure}
  
  \vspace{0.5cm}
  
  \textbf{Path Coefficients:}
  \begin{itemize}
    \item X → A: $\beta = 0.42$, $p < .001$
    \item A → B: $\beta = 0.35$, $p < .001$
    \item B → Y: $\beta = 0.38$, $p < .001$
    \item X → Y (direct): $\beta = 0.18$, $p = .032$
    \item Indirect effect: $\beta = 0.29$, 95\% CI [0.19, 0.41]
  \end{itemize}
  
  \alert{61\% of total effect mediated by A→B pathway}
\end{frame}

%==============================================
% CHAPTER 4: STUDY 3
%==============================================

\section{Chapter 4: Study 3}

\begin{frame}{Study 3: Overview}
  \begin{alertblock}{Research Question}
    Can we establish causality by experimentally manipulating X, and does the effect generalize across contexts?
  \end{alertblock}
  
  \vspace{0.5cm}
  
  \textbf{Motivation:}
  \begin{itemize}
    \item Studies 1-2 showed correlational evidence
    \item Need experimental test for causality
    \item Test generalizability to applied context
    \item Examine boundary conditions
  \end{itemize}
  
  \vspace{0.5cm}
  
  \textbf{Design:}
  \begin{itemize}
    \item 2 (X: low vs. high) × 2 (Context: lab vs. field) factorial
    \item N = 180 (45 per condition)
    \item Random assignment to conditions
    \item Outcome: Y measured post-manipulation
  \end{itemize}
\end{frame}

\begin{frame}{Study 3: Methods}
  \textbf{Experimental Manipulation:}
  \begin{itemize}
    \item \textbf{Low X condition:} Control procedure
    \item \textbf{High X condition:} Experimental manipulation designed to increase X
    \item Manipulation check: Successful ($t(178) = 8.92$, $p < .001$, $d = 1.34$)
  \end{itemize}
  
  \vspace{0.5cm}
  
  \textbf{Contexts:}
  \begin{itemize}
    \item \textbf{Lab context:} Controlled laboratory setting (original)
    \item \textbf{Field context:} Applied real-world setting (generalization test)
  \end{itemize}
  
  \vspace{0.5cm}
  
  \textbf{Measures:}
  \begin{itemize}
    \item Primary outcome Y (same as Studies 1-2)
    \item Mediators A and B
    \item Moderator Z
    \item Potential confounds
  \end{itemize}
\end{frame}

\begin{frame}{Study 3: Results}
  \begin{columns}[T]
    
    \begin{column}{0.5\textwidth}
      \begin{figure}
        \centering
        % \includegraphics[width=\textwidth]{study3_results.pdf}
        \framebox[0.9\textwidth][c]{[Experimental Results]}
        \caption{Main effect of X on Y}
      \end{figure}
    \end{column}
    
    \begin{column}{0.5\textwidth}
      \textbf{ANOVA Results:}
      \begin{itemize}
        \item Main effect of X: $F(1,176) = 45.2$, $p < .001$, $\eta^2_p = .20$
        \item Main effect of Context: $F(1,176) = 2.1$, $p = .15$
        \item X × Context: $F(1,176) = 0.8$, $p = .38$
      \end{itemize}
      
      \vspace{0.5cm}
      
      \begin{block}{Key Finding}
        Causal effect of X on Y confirmed; generalizes across contexts
      \end{block}
    \end{column}
    
  \end{columns}
  
  \vspace{0.5cm}
  
  \textbf{Mediation:} Experimental mediation analysis confirmed A and B as mechanisms
\end{frame}

%==============================================
% CHAPTER 5: GENERAL DISCUSSION
%==============================================

\section{Chapter 5: General Discussion}

\begin{frame}{Synthesis Across Studies}
  \begin{table}
    \centering
    \caption{Summary of findings across three studies}
    \small
    \begin{tabular}{lccc}
      \toprule
      \textbf{Finding} & \textbf{Study 1} & \textbf{Study 2} & \textbf{Study 3} \\
      \midrule
      X → Y relationship & Yes & Yes & Yes (causal) \\
      Mediation by A & --- & Yes & Yes \\
      Mediation by B & --- & Yes & Yes \\
      Moderation by Z & Yes & Yes & Yes \\
      Generalization & --- & --- & Yes \\
      \bottomrule
    \end{tabular}
  \end{table}
  
  \vspace{0.5cm}
  
  \textbf{Convergent Evidence:}
  \begin{itemize}
    \item Robust X→Y relationship across designs and samples
    \item Consistent mediation by A→B pathway
    \item Moderation by Z replicated
    \item Effects generalize from lab to field
  \end{itemize}
\end{frame}

\begin{frame}{Theoretical Contributions}
  \begin{exampleblock}{Novel Theoretical Framework}
    This dissertation proposes and validates the XYZ Model, which integrates constructs A, B, and C to explain how X influences Y.
  \end{exampleblock}
  
  \vspace{0.5cm}
  
  \textbf{Specific Contributions:}
  \begin{enumerate}
    \item \textbf{Integration:} Bridges previously separate literatures on A and B
    \item \textbf{Mechanism:} Identifies A→B as key mediating pathway
    \item \textbf{Boundary conditions:} Specifies role of moderator Z
    \item \textbf{Generalizability:} Shows effects across contexts
    \item \textbf{Causality:} Establishes X as causal factor
  \end{enumerate}
  
  \vspace{0.5cm}
  
  \textbf{Advances Beyond Prior Work:}
  \begin{itemize}
    \item More comprehensive than Theory 1 \cite{theory1}
    \item Resolves contradictions between Studies A and B
    \item Provides testable predictions for future research
  \end{itemize}
\end{frame}

\begin{frame}{Practical Implications}
  \begin{columns}[T]
    
    \begin{column}{0.5\textwidth}
      \textbf{Clinical Applications:}
      \begin{itemize}
        \item Assessment: Screen for X
        \item Intervention target: Increase A and B
        \item Tailoring: Consider moderator Z
        \item Outcome: Expect improvement in Y
      \end{itemize}
      
      \vspace{0.5cm}
      
      \textbf{Implementation:}
      \begin{itemize}
        \item Feasibility demonstrated in field study
        \item Scalable to larger populations
        \item Cost-effective approach
      \end{itemize}
    \end{column}
    
    \begin{column}{0.5\textwidth}
      \textbf{Policy Recommendations:}
      \begin{enumerate}
        \item Support programs targeting X
        \item Fund interventions enhancing A
        \item Consider individual differences Z
        \item Monitor outcomes Y
      \end{enumerate}
      
      \vspace{0.5cm}
      
      \begin{alertblock}{Impact}
        Findings suggest potential to improve outcomes for population experiencing low X
      \end{alertblock}
    \end{column}
    
  \end{columns}
\end{frame}

\begin{frame}{Limitations and Future Directions}
  \textbf{Study Limitations:}
  \begin{enumerate}
    \item \textbf{Sample:} Primarily university-educated, young adults
    \begin{itemize}
      \item Future: Community samples, diverse populations
    \end{itemize}
    
    \item \textbf{Measures:} Some reliance on self-report
    \begin{itemize}
      \item Future: Multi-method assessment (behavioral, biological)
    \end{itemize}
    
    \item \textbf{Time frame:} Longest follow-up was 12 months
    \begin{itemize}
      \item Future: Longer-term longitudinal studies
    \end{itemize}
    
    \item \textbf{Mechanisms:} Other pathways may exist
    \begin{itemize}
      \item Future: Explore alternative mediators
    \end{itemize}
  \end{enumerate}
\end{frame}

\begin{frame}{Future Research Program}
  \begin{block}{Immediate Next Steps}
    \begin{itemize}
      \item Replicate in clinical populations
      \item Develop intervention based on findings
      \item Test with diverse samples
      \item Examine individual differences in response
    \end{itemize}
  \end{block}
  
  \vspace{0.5cm}
  
  \textbf{Long-Term Research Agenda:}
  \begin{enumerate}
    \item \textbf{Mechanism refinement:} Neural/biological underpinnings
    \item \textbf{Intervention development:} RCT of theory-driven treatment
    \item \textbf{Moderator exploration:} Genetic, environmental factors
    \item \textbf{Translation:} Dissemination and implementation science
    \item \textbf{Extension:} Apply framework to related phenomena
  \end{enumerate}
  
  \vspace{0.5cm}
  
  \textbf{Collaboration Opportunities:}
  \begin{itemize}
    \item Clinical partners for intervention trials
    \item Neuroscientists for mechanism studies
    \item Community organizations for implementation
  \end{itemize}
\end{frame}

%==============================================
% CONCLUSIONS
%==============================================

\section{Conclusions}

\begin{frame}{Dissertation Conclusions}
  \begin{exampleblock}{Central Thesis (Revisited)}
    Through three complementary studies, this dissertation demonstrates that X influences Y through mechanisms A and B, moderated by Z, with effects generalizing across contexts.
  \end{exampleblock}
  
  \vspace{0.5cm}
  
  \textbf{Key Achievements:}
  \begin{enumerate}
    \item Established robust X→Y relationship across designs
    \item Identified and validated A→B mediating pathway
    \item Demonstrated causality via experimental manipulation
    \item Showed generalizability from lab to field
    \item Proposed novel XYZ theoretical framework
  \end{enumerate}
  
  \vspace{0.5cm}
  
  \textbf{Significance:}
  \begin{itemize}
    \item Theoretical advancement in understanding X→Y processes
    \item Methodological contribution through multi-study design
    \item Practical applications for intervention and policy
    \item Foundation for sustained research program
  \end{itemize}
\end{frame}

\begin{frame}{Final Thoughts}
  \begin{block}{Take-Home Message}
    This dissertation provides compelling converging evidence that X causes Y through mechanisms A and B, offering both theoretical understanding and practical pathways for intervention.
  \end{block}
  
  \vspace{1cm}
  
  \textbf{Broader Impact:}
  \begin{itemize}
    \item Advances scientific understanding of fundamental process
    \item Provides evidence-based framework for practitioners
    \item Opens new avenues for future research
    \item Demonstrates potential to improve outcomes for affected populations
  \end{itemize}
  
  \vspace{1cm}
  
  \begin{center}
    \textit{"The best way to predict the future is to create it."} \\
    -- Peter Drucker
  \end{center}
\end{frame}

\begin{frame}[plain]
  \begin{center}
    {\LARGE \textbf{Thank You}}
    
    \vspace{1cm}
    
    {\Large Questions from the Committee}
    
    \vspace{1.5cm}
    
    \textbf{Your Name, M.S.}\\
    Doctoral Candidate\\
    Department of Your Field\\
    University Name\\
    \texttt{yourname@university.edu}
    
    \vspace{1cm}
    
    {\footnotesize
      \textbf{Funding Acknowledgment:}\\
      This research was supported by [Grant Agency] Grant \#[Number],\\
      [Fellowship Name], and [University] Dissertation Fellowship
      
      \vspace{0.5cm}
      
      \textbf{Special Thanks:}\\
      My advisor Prof. [Name], committee members, lab colleagues,\\
      study participants, and my family for their unwavering support
    }
  \end{center}
\end{frame}

%==============================================
% BACKUP SLIDES
%==============================================

\appendix

\begin{frame}{Backup: Study 1 Full Results}
  \begin{table}
    \centering
    \caption{Complete regression results for Study 1}
    \footnotesize
    \begin{tabular}{lcccc}
      \toprule
      \textbf{Predictor} & $\boldsymbol{\beta}$ & \textbf{SE} & \textbf{$t$} & \textbf{$p$} \\
      \midrule
      \multicolumn{5}{l}{\textit{Step 1: Demographics}} \\
      Age & 0.12 & 0.04 & 3.00 & .003 \\
      Gender & 0.08 & 0.05 & 1.60 & .110 \\
      Education & 0.15 & 0.04 & 3.75 & < .001 \\
      \midrule
      \multicolumn{5}{l}{\textit{Step 2: Main Effect}} \\
      X & 0.47 & 0.04 & 11.75 & < .001 \\
      \midrule
      \multicolumn{5}{l}{\textit{Step 3: Moderation}} \\
      Z & 0.18 & 0.04 & 4.50 & < .001 \\
      X × Z & 0.12 & 0.04 & 3.00 & .003 \\
      \bottomrule
      \multicolumn{5}{l}{Final model: $R^2 = .28$, $F(6,493) = 32.1$, $p < .001$} \\
    \end{tabular}
  \end{table}
\end{frame}

\begin{frame}{Backup: Study 2 Model Fit}
  \textbf{Structural Equation Model Fit Indices:}
  
  \begin{table}
    \centering
    \begin{tabular}{lcc}
      \toprule
      \textbf{Index} & \textbf{Value} & \textbf{Criterion} \\
      \midrule
      $\chi^2$/df & 2.34 & < 3.0 \\
      CFI & 0.96 & > 0.95 \\
      TLI & 0.95 & > 0.95 \\
      RMSEA & 0.045 & < 0.06 \\
      SRMR & 0.038 & < 0.08 \\
      \bottomrule
    \end{tabular}
  \end{table}
  
  \vspace{0.5cm}
  
  \textbf{Conclusion:} Excellent model fit, proposed model fits data well
  
  \vspace{0.5cm}
  
  \textbf{Alternative Models Tested:}
  \begin{itemize}
    \item Direct-only model: $\Delta\chi^2(2) = 45.6$, $p < .001$ (worse fit)
    \item Reverse mediation: $\Delta\chi^2(2) = 38.2$, $p < .001$ (worse fit)
    \item Proposed model provides best fit
  \end{itemize}
\end{frame}

\begin{frame}{Backup: Study 3 Additional Analyses}
  \textbf{Subgroup Effects:}
  
  \begin{figure}
    \centering
    % \includegraphics[width=0.7\textwidth]{study3_subgroups.pdf}
    \framebox[0.65\textwidth][c]{[Subgroup Analysis Results]}
    \caption{Effect of X on Y by moderator Z levels}
  \end{figure}
  
  \begin{itemize}
    \item High Z: $d = 0.95$, $p < .001$
    \item Medium Z: $d = 0.72$, $p < .001$
    \item Low Z: $d = 0.45$, $p = .008$
    \item Moderation: $F(2,174) = 6.8$, $p = .001$
  \end{itemize}
\end{frame}

%==============================================
% REFERENCES
%==============================================

\begin{frame}[allowframebreaks]{References}
  \printbibliography
\end{frame}

\end{document}

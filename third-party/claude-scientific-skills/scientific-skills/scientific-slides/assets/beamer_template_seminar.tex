\documentclass[aspectratio=169,11pt]{beamer}

% Encoding
\usepackage[utf8]{inputenc}
\usepackage[T1]{fontenc}

% Theme and colors
\usetheme{Madrid}
\usecolortheme{dolphin}

% Remove navigation symbols
\setbeamertemplate{navigation symbols}{}

% Section pages
\AtBeginSection[]{
  \begin{frame}
    \vfill
    \centering
    \begin{beamercolorbox}[sep=8pt,center,shadow=true,rounded=true]{title}
      \usebeamerfont{title}\insertsectionhead\par%
    \end{beamercolorbox}
    \vfill
  \end{frame}
}

% Graphics
\usepackage{graphicx}
\graphicspath{{./figures/}}

% Math
\usepackage{amsmath, amssymb, amsthm}

% Tables
\usepackage{booktabs}
\usepackage{multirow}

% Citations
\usepackage[style=authoryear,maxcitenames=2,backend=biber]{biblatex}
\addbibresource{references.bib}
\renewcommand*{\bibfont}{\tiny}

% Algorithms
\usepackage{algorithm}
\usepackage{algorithmic}

% Code
\usepackage{listings}
\lstset{
  basicstyle=\ttfamily\small,
  keywordstyle=\color{blue},
  commentstyle=\color{green!60!black},
  stringstyle=\color{orange},
  numbers=left,
  numberstyle=\tiny,
  frame=single,
  breaklines=true
}

% Custom colors
\definecolor{darkblue}{RGB}{0,75,135}
\definecolor{lightblue}{RGB}{100,150,200}

\setbeamercolor{structure}{fg=darkblue}
\setbeamercolor{title}{fg=darkblue}
\setbeamercolor{frametitle}{fg=darkblue}

% Title information
\title[Short Title for Footer]{Full Title of Your Research:\\Comprehensive and Descriptive}
\subtitle{Research Seminar Presentation}
\author[Your Name]{Your Name, PhD Candidate\\
  Advisor: Prof. Advisor Name}
\institute[University]{
  Department of Your Field\\
  University Name\\
  \vspace{0.2cm}
  \texttt{yourname@university.edu}
}
\date{\today}

% Logo (optional)
% \logo{\includegraphics[height=0.8cm]{university_logo.png}}

\begin{document}

% Title slide
\begin{frame}[plain]
  \titlepage
\end{frame}

% Outline
\begin{frame}{Outline}
  \tableofcontents
\end{frame}

%==============================================
% INTRODUCTION
%==============================================

\section{Introduction}

\begin{frame}{Motivation}
  \begin{columns}[T]
    
    \begin{column}{0.5\textwidth}
      \textbf{The Big Picture:}
      \begin{itemize}
        \item Why this research area matters
        \item Real-world impact and applications
        \item Current challenges in the field
        \item Opportunity for advancement
      \end{itemize}
    \end{column}
    
    \begin{column}{0.5\textwidth}
      \begin{figure}
        \centering
        % \includegraphics[width=\textwidth]{motivation_figure.pdf}
        \framebox[0.9\textwidth][c]{[Motivating Figure]}
        \caption{Illustration of the problem or impact}
      \end{figure}
    \end{column}
    
  \end{columns}
  
  \vspace{0.5cm}
  
  \begin{block}{Central Question}
    How can we address this important challenge using novel approach X?
  \end{block}
\end{frame}

\subsection{Background}

\begin{frame}{Prior Work: Overview}
  \textbf{Historical Development:}
  \begin{itemize}
    \item Early work established foundation \cite{seminal1990}
    \item Key advances in 2000s \cite{advance2005,advance2007}
    \item Recent developments \cite{recent2020,recent2022}
  \end{itemize}
  
  \vspace{0.5cm}
  
  \textbf{Current State of Knowledge:}
  \begin{enumerate}
    \item We know that X affects Y
    \item Evidence suggests mechanism involves Z
    \item However, questions remain about W
  \end{enumerate}
\end{frame}

\begin{frame}{Knowledge Gap}
  \begin{columns}[c]
    
    \begin{column}{0.6\textwidth}
      \textbf{What We Know:}
      \begin{itemize}
        \item Point 1: Established finding
        \item Point 2: Replicated result
        \item Point 3: General consensus
      \end{itemize}
      
      \vspace{0.5cm}
      
      \textbf{What Remains Unknown:}
      \begin{itemize}
        \item \alert{Gap 1:} Critical unknown
        \item \alert{Gap 2:} Methodological limitation
        \item \alert{Gap 3:} Unexplored context
      \end{itemize}
    \end{column}
    
    \begin{column}{0.4\textwidth}
      \begin{alertblock}{The Problem}
        Existing approaches fail to account for X, limiting our understanding of Y and preventing application to Z.
      \end{alertblock}
    \end{column}
    
  \end{columns}
\end{frame}

\subsection{Research Questions}

\begin{frame}{Research Objectives}
  \begin{exampleblock}{Overall Goal}
    To investigate how X influences Y under conditions Z, and develop a framework for understanding mechanism W.
  \end{exampleblock}
  
  \vspace{0.5cm}
  
  \textbf{Specific Aims:}
  \begin{enumerate}
    \item \textbf{Aim 1:} Characterize relationship between X and Y
    \begin{itemize}
      \item Hypothesis: X positively correlates with Y
    \end{itemize}
    
    \item \textbf{Aim 2:} Identify mechanism W mediating X→Y
    \begin{itemize}
      \item Hypothesis: W explains the X-Y relationship
    \end{itemize}
    
    \item \textbf{Aim 3:} Test generalizability to context Z
    \begin{itemize}
      \item Hypothesis: Effect persists across conditions
    \end{itemize}
  \end{enumerate}
\end{frame}

%==============================================
% METHODS
%==============================================

\section{Methods}

\subsection{Study Design}

\begin{frame}{Overall Approach}
  \begin{figure}
    \centering
    % \includegraphics[width=0.9\textwidth]{study_design.pdf}
    \framebox[0.8\textwidth][c]{[Study Design Schematic]}
    \caption{Three-phase experimental design}
  \end{figure}
  
  \begin{itemize}
    \item \textbf{Phase 1:} Observational study (n = 150)
    \item \textbf{Phase 2:} Controlled experiment (n = 80)
    \item \textbf{Phase 3:} Validation in new context (n = 120)
  \end{itemize}
\end{frame}

\subsection{Participants and Materials}

\begin{frame}{Sample Characteristics}
  \begin{columns}[T]
    
    \begin{column}{0.5\textwidth}
      \textbf{Inclusion Criteria:}
      \begin{itemize}
        \item Age 18-65 years
        \item Criterion 2
        \item Criterion 3
      \end{itemize}
      
      \vspace{0.3cm}
      
      \textbf{Exclusion Criteria:}
      \begin{itemize}
        \item Confound 1
        \item Confound 2
      \end{itemize}
    \end{column}
    
    \begin{column}{0.5\textwidth}
      \begin{table}
        \centering
        \caption{Sample demographics}
        \small
        \begin{tabular}{lc}
          \toprule
          \textbf{Variable} & \textbf{Value} \\
          \midrule
          N & 150 \\
          Age (years) & 32.5 $\pm$ 8.2 \\
          Female (\%) & 58 \\
          Education (years) & 15.2 $\pm$ 2.1 \\
          \bottomrule
        \end{tabular}
      \end{table}
    \end{column}
    
  \end{columns}
  
  \vspace{0.3cm}
  
  \footnotesize Recruitment: University community and online platforms
\end{frame}

\subsection{Procedures}

\begin{frame}{Experimental Procedure}
  \begin{columns}[T]
    
    \begin{column}{0.5\textwidth}
      \textbf{Session 1 (60 min):}
      \begin{enumerate}
        \item Informed consent
        \item Baseline measures
        \item Training phase (20 min)
        \item Test phase (30 min)
      \end{enumerate}
      
      \vspace{0.5cm}
      
      \textbf{Session 2 (45 min):}
      \begin{enumerate}
        \setcounter{enumi}{4}
        \item Follow-up measures
        \item Manipulation (15 min)
        \item Final assessment (25 min)
      \end{enumerate}
    \end{column}
    
    \begin{column}{0.5\textwidth}
      \begin{figure}
        \centering
        % \includegraphics[width=\textwidth]{procedure_timeline.pdf}
        \framebox[0.9\textwidth][c]{[Timeline Diagram]}
        \caption{Experimental timeline}
      \end{figure}
      
      \vspace{0.5cm}
      
      \begin{alertblock}{Key Innovation}
        Novel manipulation technique combining approach A with method B
      \end{alertblock}
    \end{column}
    
  \end{columns}
\end{frame}

\subsection{Analysis}

\begin{frame}{Statistical Analysis Plan}
  \textbf{Primary Analyses:}
  \begin{itemize}
    \item \textbf{Aim 1:} Linear regression: $Y = \beta_0 + \beta_1 X + \epsilon$
    \item \textbf{Aim 2:} Mediation analysis using bootstrapping (5000 iterations)
    \item \textbf{Aim 3:} Mixed-effects model accounting for context effects
  \end{itemize}
  
  \vspace{0.5cm}
  
  \textbf{Secondary Analyses:}
  \begin{itemize}
    \item Sensitivity analyses with different covariates
    \item Subgroup analyses by demographic factors
    \item Exploratory analyses of individual differences
  \end{itemize}
  
  \vspace{0.5cm}
  
  \begin{block}{Software}
    R 4.2.1 (lme4, lavaan packages); Python 3.10 (scikit-learn); SPSS 28
  \end{block}
\end{frame}

%==============================================
% RESULTS
%==============================================

\section{Results}

\subsection{Preliminary Analyses}

\begin{frame}{Data Quality and Assumptions}
  \begin{columns}[T]
    
    \begin{column}{0.5\textwidth}
      \textbf{Data Screening:}
      \begin{itemize}
        \item Missing data: < 5\% per variable
        \item Outliers: 3 cases removed
        \item Assumptions: All met
      \end{itemize}
      
      \vspace{0.3cm}
      
      \textbf{Descriptive Statistics:}
      \begin{itemize}
        \item Variable X: $M = 45.2$, $SD = 8.1$
        \item Variable Y: $M = 67.8$, $SD = 12.3$
        \item Correlation: $r = 0.54$, $p < .001$
      \end{itemize}
    \end{column}
    
    \begin{column}{0.5\textwidth}
      \begin{figure}
        \centering
        % \includegraphics[width=\textwidth]{descriptives.pdf}
        \framebox[0.9\textwidth][c]{[Descriptive Plots]}
        \caption{Variable distributions}
      \end{figure}
    \end{column}
    
  \end{columns}
\end{frame}

\subsection{Aim 1 Results}

\begin{frame}{Aim 1: X Predicts Y}
  \begin{columns}[c]
    
    \begin{column}{0.6\textwidth}
      \begin{figure}
        \centering
        % \includegraphics[width=\textwidth]{aim1_result.pdf}
        \framebox[0.9\textwidth][c]{[Regression Plot]}
        \caption{Relationship between X and Y ($R^2 = 0.29$, $p < .001$)}
      \end{figure}
    \end{column}
    
    \begin{column}{0.4\textwidth}
      \begin{table}
        \centering
        \caption{Regression results}
        \tiny
        \begin{tabular}{lccc}
          \toprule
          \textbf{Predictor} & $\boldsymbol{\beta}$ & \textbf{SE} & \textbf{$p$} \\
          \midrule
          Intercept & 12.45 & 3.21 & < .001 \\
          X & 0.54 & 0.08 & < .001 \\
          Age & 0.12 & 0.05 & .018 \\
          Gender & 2.34 & 1.12 & .038 \\
          \bottomrule
        \end{tabular}
      \end{table}
      
      \vspace{0.3cm}
      
      \begin{block}{Key Finding}
        X significantly predicts Y, controlling for demographics
      \end{block}
    \end{column}
    
  \end{columns}
\end{frame}

\subsection{Aim 2 Results}

\begin{frame}{Aim 2: Mediation by W}
  \begin{figure}
    \centering
    % \includegraphics[width=0.8\textwidth]{mediation_model.pdf}
    \framebox[0.7\textwidth][c]{[Mediation Diagram]}
    \caption{Mediation analysis showing W mediates X→Y relationship}
  \end{figure}
  
  \begin{itemize}
    \item \textbf{Direct effect:} $c' = 0.31$, $p = .021$ (reduced from $c = 0.54$)
    \item \textbf{Indirect effect:} $ab = 0.23$, 95\% CI [0.14, 0.35]
    \item \textbf{Proportion mediated:} 43\% of total effect
  \end{itemize}
  
  \vspace{0.3cm}
  
  \alert{W partially mediates the relationship between X and Y}
\end{frame}

\subsection{Aim 3 Results}

\begin{frame}{Aim 3: Generalization to Context Z}
  \begin{columns}[T]
    
    \begin{column}{0.5\textwidth}
      \begin{figure}
        \centering
        % \includegraphics[width=\textwidth]{aim3_context1.pdf}
        \framebox[0.9\textwidth][c]{[Context 1]}
        \caption{Original context}
      \end{figure}
    \end{column}
    
    \begin{column}{0.5\textwidth}
      \begin{figure}
        \centering
        % \includegraphics[width=\textwidth]{aim3_context2.pdf}
        \framebox[0.9\textwidth][c]{[Context 2]}
        \caption{New context Z}
      \end{figure}
    \end{column}
    
  \end{columns}
  
  \vspace{0.5cm}
  
  \textbf{Mixed-Effects Model Results:}
  \begin{itemize}
    \item Main effect of X: $\beta = 0.51$, $p < .001$
    \item Context × X interaction: $\beta = -0.08$, $p = .231$ (ns)
    \item \alert{Effect generalizes across contexts}
  \end{itemize}
\end{frame}

\subsection{Additional Analyses}

\begin{frame}{Sensitivity and Robustness Checks}
  \textbf{Alternative Specifications:}
  \begin{itemize}
    \item Result robust to different model specifications
    \item Consistent across multiple imputation methods
    \item Findings hold with/without covariates
  \end{itemize}
  
  \vspace{0.5cm}
  
  \textbf{Subgroup Analyses:}
  \begin{table}
    \centering
    \caption{Effect sizes by subgroup}
    \small
    \begin{tabular}{lccc}
      \toprule
      \textbf{Subgroup} & \textbf{$n$} & $\boldsymbol{\beta}$ & \textbf{$p$} \\
      \midrule
      Young (< 30) & 67 & 0.58 & < .001 \\
      Older ($\geq$ 30) & 83 & 0.49 & < .001 \\
      Male & 63 & 0.52 & < .001 \\
      Female & 87 & 0.55 & < .001 \\
      \bottomrule
    \end{tabular}
  \end{table}
  
  Effect consistent across demographic groups
\end{frame}

%==============================================
% DISCUSSION
%==============================================

\section{Discussion}

\subsection{Summary of Findings}

\begin{frame}{Key Results Recap}
  \begin{exampleblock}{Main Findings}
    \begin{enumerate}
      \item X significantly predicts Y ($\beta = 0.54$, $p < .001$), explaining 29\% of variance
      \item W mediates 43\% of the X→Y relationship
      \item Effect generalizes to new context Z
      \item Results robust across subgroups and specifications
    \end{enumerate}
  \end{exampleblock}
  
  \vspace{0.5cm}
  
  \textbf{These findings:}
  \begin{itemize}
    \item Support our hypotheses
    \item Provide evidence for mechanism W
    \item Extend previous work to new domains
    \item Have implications for theory and practice
  \end{itemize}
\end{frame}

\subsection{Interpretation}

\begin{frame}{Relation to Previous Research}
  \begin{columns}[T]
    
    \begin{column}{0.5\textwidth}
      \textbf{Consistent With:}
      \begin{itemize}
        \item Prior findings on X→Y \cite{jones2020}
        \item Theoretical predictions \cite{smith2019}
        \item Meta-analytic trends \cite{meta2021}
      \end{itemize}
      
      \vspace{0.5cm}
      
      \textbf{Extensions Beyond:}
      \begin{itemize}
        \item Identifies mechanism W (new)
        \item Tests in context Z (novel)
        \item Larger sample than prior work
      \end{itemize}
    \end{column}
    
    \begin{column}{0.5\textwidth}
      \textbf{Resolves Contradictions:}
      \begin{itemize}
        \item Explains why Study A found X
        \item Reconciles Studies B and C
        \item Clarifies conditions for effect
      \end{itemize}
      
      \vspace{0.5cm}
      
      \begin{alertblock}{Novel Contribution}
        First study to demonstrate W as mediator and show generalization to Z
      \end{alertblock}
    \end{column}
    
  \end{columns}
\end{frame}

\begin{frame}{Mechanisms and Explanations}
  \textbf{Why does X affect Y through W?}
  
  \vspace{0.3cm}
  
  \begin{enumerate}
    \item<1-> \textbf{Hypothesis 1:} X activates process W
    \begin{itemize}
      \item<1-> Evidence: Temporal precedence in data
      \item<1-> Consistent with neurobiological models
    \end{itemize}
    
    \vspace{0.3cm}
    
    \item<2-> \textbf{Hypothesis 2:} W is necessary for Y
    \begin{itemize}
      \item<2-> Evidence: Mediation analysis results
      \item<2-> Supported by experimental manipulations
    \end{itemize}
    
    \vspace{0.3cm}
    
    \item<3-> \textbf{Integrated Model:} X → W → Y pathway
    \begin{itemize}
      \item<3-> Explains 43\% of total effect
      \item<3-> Other pathways remain to be identified
    \end{itemize}
  \end{enumerate}
\end{frame}

\subsection{Implications}

\begin{frame}{Theoretical Implications}
  \textbf{Advances to Theory:}
  \begin{itemize}
    \item Refines existing framework by identifying W
    \item Suggests revision of Model XYZ
    \item Provides testable predictions for future work
    \item Integrates previously separate literatures
  \end{itemize}
  
  \vspace{0.5cm}
  
  \textbf{Broader Scientific Impact:}
  \begin{itemize}
    \item Methodology can be applied to related domains
    \item Framework generalizable to other contexts
    \item Opens new research directions
  \end{itemize}
\end{frame}

\begin{frame}{Practical Applications}
  \begin{columns}[T]
    
    \begin{column}{0.5\textwidth}
      \textbf{Clinical/Applied:}
      \begin{itemize}
        \item Intervention target: W
        \item Assessment tool: Measure X
        \item Treatment planning: Consider Z
        \item Expected benefit: Improvement in Y
      \end{itemize}
    \end{column}
    
    \begin{column}{0.5\textwidth}
      \textbf{Policy Implications:}
      \begin{itemize}
        \item Recommendation 1
        \item Recommendation 2
        \item Implementation considerations
        \item Cost-benefit analysis
      \end{itemize}
    \end{column}
    
  \end{columns}
  
  \vspace{0.5cm}
  
  \begin{exampleblock}{Translational Path}
    Findings suggest feasibility of intervention targeting W to improve Y in population experiencing X
  \end{exampleblock}
\end{frame}

\subsection{Limitations and Future Directions}

\begin{frame}{Limitations}
  \textbf{Study Limitations:}
  \begin{enumerate}
    \item \textbf{Cross-sectional design}: Cannot establish causality definitively
    \begin{itemize}
      \item Future: Longitudinal or experimental design
    \end{itemize}
    
    \item \textbf{Sample characteristics}: University students, may limit generalizability
    \begin{itemize}
      \item Future: Community sample, diverse populations
    \end{itemize}
    
    \item \textbf{Measurement}: Self-report bias possible for some variables
    \begin{itemize}
      \item Future: Incorporate objective measures
    \end{itemize}
    
    \item \textbf{Unmeasured confounds}: Other factors could explain relationships
    \begin{itemize}
      \item Future: Control for additional variables
    \end{itemize}
  \end{enumerate}
\end{frame}

\begin{frame}{Future Research Directions}
  \begin{block}{Immediate Next Steps}
    \begin{itemize}
      \item Replicate in independent sample
      \item Test causal model experimentally
      \item Examine boundary conditions
    \end{itemize}
  \end{block}
  
  \vspace{0.5cm}
  
  \textbf{Longer-Term Goals:}
  \begin{itemize}
    \item Develop intervention based on findings
    \item Investigate neural mechanisms
    \item Explore individual differences
    \item Translate to applied settings
  \end{itemize}
  
  \vspace{0.5cm}
  
  \textbf{Collaborations Sought:}
  \begin{itemize}
    \item Experts in domain A for validation
    \item Clinical partners for translation
    \item Methodologists for advanced analyses
  \end{itemize}
\end{frame}

%==============================================
% CONCLUSION
%==============================================

\section{Conclusion}

\begin{frame}{Conclusions}
  \begin{exampleblock}{Key Contributions}
    \begin{enumerate}
      \item Demonstrated robust X→Y relationship
      \item Identified W as mediating mechanism
      \item Showed generalizability across contexts
      \item Provided framework for future research
    \end{enumerate}
  \end{exampleblock}
  
  \vspace{0.5cm}
  
  \begin{block}{Take-Home Message}
    Our findings reveal that X influences Y through mechanism W, providing new understanding of this important process and suggesting potential intervention targets.
  \end{block}
  
  \vspace{0.5cm}
  
  \textbf{Impact:}
  \begin{itemize}
    \item Theoretical advancement in understanding X→Y
    \item Practical implications for interventions
    \item Foundation for future research program
  \end{itemize}
\end{frame}

\begin{frame}[plain]
  \begin{center}
    {\LARGE \textbf{Thank You}}
    
    \vspace{1cm}
    
    {\Large Questions \& Discussion}
    
    \vspace{1.5cm}
    
    \begin{columns}
      \begin{column}{0.6\textwidth}
        \textbf{Contact Information:}\\
        Your Name\\
        Department of Your Field\\
        University Name\\
        \texttt{yourname@university.edu}\\
        \url{https://yourlab.university.edu}
      \end{column}
      
      \begin{column}{0.4\textwidth}
        % QR code to lab website or paper
        % \includegraphics[width=4cm]{qrcode_website.png}\\
        % {\small Scan for more info}
      \end{column}
    \end{columns}
    
    \vspace{1cm}
    
    {\footnotesize
      \textbf{Acknowledgments:}\\
      Funding: NSF Grant \#12345, NIH Grant R01-67890\\
      Lab Members: Person A, Person B, Person C\\
      Collaborators: Prof. X (University Y), Dr. Z (Institution W)
    }
  \end{center}
\end{frame}

%==============================================
% BACKUP SLIDES
%==============================================

\appendix

\begin{frame}{Backup: Full Regression Table}
  \begin{table}
    \centering
    \caption{Complete regression results with all covariates}
    \footnotesize
    \begin{tabular}{lcccc}
      \toprule
      \textbf{Predictor} & $\boldsymbol{\beta}$ & \textbf{SE} & \textbf{$t$} & \textbf{$p$} \\
      \midrule
      Intercept & 12.45 & 3.21 & 3.88 & < .001 \\
      X (primary predictor) & 0.54 & 0.08 & 6.75 & < .001 \\
      Age & 0.12 & 0.05 & 2.40 & .018 \\
      Gender (female) & 2.34 & 1.12 & 2.09 & .038 \\
      Education & 0.45 & 0.31 & 1.45 & .149 \\
      Covariate Z & -0.18 & 0.09 & -2.00 & .047 \\
      \midrule
      $R^2$ & \multicolumn{4}{c}{0.35} \\
      Adjusted $R^2$ & \multicolumn{4}{c}{0.32} \\
      $F$(5,144) & \multicolumn{4}{c}{15.48, $p < .001$} \\
      \bottomrule
    \end{tabular}
  \end{table}
\end{frame}

\begin{frame}{Backup: Alternative Analysis}
  \begin{figure}
    \centering
    % \includegraphics[width=0.75\textwidth]{sensitivity_analysis.pdf}
    \framebox[0.7\textwidth][c]{[Sensitivity Analysis Results]}
    \caption{Results robust across different model specifications}
  \end{figure}
\end{frame}

\begin{frame}{Backup: Detailed Methods}
  \textbf{Measurement Details:}
  \begin{itemize}
    \item \textbf{Variable X:} Scale name (Author, Year)
    \begin{itemize}
      \item 12 items, 5-point Likert scale
      \item Cronbach's $\alpha = 0.89$
      \item Example item: "Statement here"
    \end{itemize}
    
    \item \textbf{Variable Y:} Assessment tool
    \begin{itemize}
      \item Performance-based measure
      \item Inter-rater reliability: ICC = 0.92
      \item Range: 0-100
    \end{itemize}
    
    \item \textbf{Mediator W:} Experimental manipulation check
    \begin{itemize}
      \item Manipulation successful: $t(149) = 8.45$, $p < .001$
      \item Effect size: $d = 1.38$
    \end{itemize}
  \end{itemize}
\end{frame}

%==============================================
% REFERENCES
%==============================================

\begin{frame}[allowframebreaks]{References}
  \printbibliography
\end{frame}

\end{document}
